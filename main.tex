\documentclass{llncs}
\usepackage{makeidx}
\usepackage{amsmath}
\usepackage{algorithm}
\usepackage[noend]{algpseudocode}
\usepackage[utf8]{inputenc}
\usepackage{cite}
\usepackage{float}
\usepackage{caption}
\usepackage[pdftex]{graphicx}
\usepackage[nottoc]{tocbibind}
\usepackage{pdfpages}
\usepackage{ragged2e}
\usepackage{hyperref}
%
\usepackage{color}
\newcommand{\PR}[1]{{\textcolor{green}{[PR: #1]}}}  % color gei
\newcommand{\DT}[1]{{\textcolor{magenta}{[DT: #1]}}}% color gei
\newcommand{\TR}[1]{{\textcolor{cyan}{[TR: #1]}}}   % color gei
\newcommand{\RK}[1]{{\textcolor{blue}{[Rk3: #1]}}}  % color gei
\newcommand{\FP}[1]{{\textcolor{orange}{[FP: #1]}}} % color gei
\newcommand{\MA}[1]{{\textcolor{red}{[MA: #1]}}}
\newcommand{\NH}[1]{{\textcolor{purple}{[NH: #1]}}} % color the rial macho


\begin{document}
\tracingall
\mainmatter

\title{Sysmic Robotics Team Description Paper}
\titlerunning{Sysmic Robotics TDP 2019}

\author{Tom\'as Rodenas, Ricardo Alfaro,  Pablo Reyes, Felipe Pinto, Maximiliano Aubel, Nicolás Hernández, Pablo Yañez, Tania Barrera, Daniel Torres, Jorge Álvarez, Damián Quiroz, Rubén González.}
%
\tocauthor{Tom\'as Rodenas, Ricardo Alfaro,  Pablo Reyes, Felipe Pinto, Maximiliano Aubel, Nicolás Hernández, Pablo Yañez, Tania Barrera, Daniel Torres, Jorge Álvarez, Damián Quiroz, Rubén González.}
%
\institute{Universidad T\'ecnica Federico Santa Mar\'ia, Valpara\'iso, Chile}

\maketitle

%%%%%%%%%%%%%%%%%%%%%%%%%%%%%%%%%%%%%%%%%%%%%%%%%%%%%%%%%%%%%%%%%%%%%%%%%%%%%%%%%%%%%%%%%%%%%%%
%                                 Abstract                                                    %
%%%%%%%%%%%%%%%%%%%%%%%%%%%%%%%%%%%%%%%%%%%%%%%%%%%%%%%%%%%%%%%%%%%%%%%%%%%%%%%%%%%%%%%%%%%%%%%
\begin{abstract}
    In this paper we briefly describe the current state of the systems implemented by our team. \PR{Comentar de forma resumida los cambios desde TDP2019 a ahora}
\end{abstract}


%%%%%%%%%%%%%%%%%%%%%%%%%%%%%%%%%%%%%%%%%%%%%%%%%%%%%%%%%%%%%%%%%%%%%%%%%%%%%%%%%%%%%%%%%%%%%%%
%                                 Mechanics                                                   %
%%%%%%%%%%%%%%%%%%%%%%%%%%%%%%%%%%%%%%%%%%%%%%%%%%%%%%%%%%%%%%%%%%%%%%%%%%%%%%%%%%%%%%%%%%%%%%%
\section{Mechanics}

\subsection{Gear-free motor to wheel connection}

\PR{Explicar de forma técnica los cambios del sistema de conexión "motor a rueda". Incorporar una imagen modelada del nuevo diseño y altura de la rueda.}

\begin{figure}[H]
    \centering
    \includegraphics[scale=0.6]{Images/Engranaje.jpeg}
    \caption{Front and general view of gear system. }
    \label{fig:Wheel1}
\end{figure}

\begin{figure}[H]
    \centering
    \includegraphics[scale=0.6]{Images/Wheel.jpeg}
    \caption{Front and lateral view of assembled wheel. }
    \label{fig:Wheel2}
\end{figure}

%%%%%%%%%%%%%%%%%%%%%%%%%%%%%%%%%%%%%%%%%%%%%%%%%%%%%%%%%%%%%%%%%%%%%%%%%%%%%%%%%%%%%%%%%%%%%%%
%                                 Software                                                    %
%%%%%%%%%%%%%%%%%%%%%%%%%%%%%%%%%%%%%%%%%%%%%%%%%%%%%%%%%%%%%%%%%%%%%%%%%%%%%%%%%%%%%%%%%%%%%%%
\section{Software}

\subsection{Custom client}
Through the years of the initiative, we have always worked with the client created by the robojackets team to achieve participation in the competition, but due to the new needs of our robots and the increase in members belonging to the computer science career, the creation of a client that fits the characteristics of our robots was the next step to grow. Based on this, a new version has been developed based on an unfinished project created by former members of our initiative. Our current objective is focused on creating a base framework that allows easy incorporation of future higher-level research, thus creating an excellent testing ground for old and new members. Progress is as follows:

\begin{itemize}
    \item \textbf{Frontend migration to Tailwind}
    The first client version was made with Bulma CSS, a JavaScript library that allows working easily with the design intended. It worked for a while, but it was used by a previous programmers team. In order to use the last libraries and frameworks, we decided to change our CSS library to Tailwind CSS.
    \\

    Why is this library a better option than Bulma CSS? There were several points about our choice:
    \begin{enumerate}
        \item Our main point was about the knowledge that new members had about this library.
        \item This library facilitates the directory reduction in only two CSS files.
        \item It allows to write short key words within the class property in the DOM object to code quicker and better.        
    \end{enumerate}
    
    \item \textbf{Robot connection}
    \\
    \DT{Daniel Torres}
    
    \item \textbf{Use of Python as connection between our engine and the view}\\

    Our first version of the client was completely made in JavaScript. It implies that the view and the connection that allowed to show our robots digitally used JavaScript libraries. Sadly, we used \href{https://www.npmjs.com/package/dgram}{dgram}, a socket oriented library that has been deprecated no long time ago. That was a problem, because whenever it gets removed from npm, it could be a great vulnerability. \\[2pt]

    In response to this issue, we resolved to adopt Python as our intermediate language to link the engine ports with the robots information and the client ports to bind them.
    
\end{itemize}

\begin{figure}[H]
    \centering
    \includegraphics[scale=0.5]{gDiagram.png}
    \caption{General diagram of the system developed in 2019\DT{buscar referencia}}
    \label{fig:gDiagram}
\end{figure}

The core aspects of the previous version have remained unchanged, with only features already present in the overall structure being updated.

\subsection{New version of the data package}
\DT{Daniel Torres}

\PR{En el TDP2019 se especifica que la interfaz está construida en QT, mencionar los cambios en los frameworks utilizados para el diseño de la interfaz. Si hubieron cambios en el diagrama general del sistema, también mencionarlos y rehacer la figura.}


\begin{figure}[H]
    \centering
    \includegraphics[scale=0.4]{GUI_F.png}
    \caption{GUI screenshot \DT{Se necesita screenshot de la nueva interfaz}}
    \label{fig:GUI}
\end{figure}

It must be noted that this client is under construction and that we guided ourselves by the software shared by RoboJackets \cite{robojackets}. 

%%%%%%%%%%%%%%%%%%%%%%%%%%%%%%%%%%%%%%%%%%%%%%%%%%%%%%%%%%%%%%%%%%%%%%%%%%%%%%%%%%%%%%%%%%%%%%%
%                                   Sensors                                                   %
%%%%%%%%%%%%%%%%%%%%%%%%%%%%%%%%%%%%%%%%%%%%%%%%%%%%%%%%%%%%%%%%%%%%%%%%%%%%%%%%%%%%%%%%%%%%%%%
\subsection{Sensor Integration}
In the new model we integrate different kind of sensors, meaning that the robots will no longer be a peripheral to take active part in the designed AI system. The goal is to get a better estimate of the game state to enable better decision making at software level improving the game strategy, as well as at hardware level enabling a better control of the robots. Specifically we add encoders, accelerometer, gyroscope and proximity sensors.

Since the omnidirectional robots represent a nonlinear system we implement an Extended Kalman Filter  to make use of the information coming from the different kind of sensors. The main function of this algorithm is to reduce the error coming from the uncertainty present in the environment and in sensor measurement.

The encoders, accelerometer and gyroscope let us get a quicker estimate of the robot pose and velocity, since these components have a high measurement rate. However the downside of this approach is that the error propagates with the robot movement because it is relative to the robot local pose. To correct this increasing error we use the slower measurements coming from the camera taking advantage of its absolute error (since it does not depend of the robot pose and, thus, it does not propagate with it).    

Another important aspect relates to track the ball's position where we mainly use the camera information, but because of the fast nature of the game, there are significant periods of time where the robot does not know where the ball is. For that reason we implement proximity sensors to aid in the ball location when precision is needed. \PR{Se puede hacer algo al respecto de lo que está escrito?}

%%%%%%%%%%%%%%%%%%%%%%%%%%%%%%%%%%%%%%%%%%%%%%%%%%%%%%%%%%%%%%%%%%%%%%%%%%%%%%%%%%%%%%%%%%%%%%%
%                               HARDWARE
%
%%%%%%%%%%%%%%%%%%%%%%%%%%%%%%%%%%%%%%%%%%%%%%%%%%%%%%%%%%%%%%%%%%%%%%%%%%%%%%%%%%%%%%%%%%%%%%
\section{Hardware}

%%%%%%%%%%%%%%%%%%%%%%%%%%%%%%%%%%%%%%%%%%%%%%%%%%%%%%%%%%%%%%%%%%%%%%%%%%%%%%%%%%%%%%%%%%%%%%%
%                                 Kicker                                                      %
%%%%%%%%%%%%%%%%%%%%%%%%%%%%%%%%%%%%%%%%%%%%%%%%%%%%%%%%%%%%%%%%%%%%%%%%%%%%%%%%%%%%%%%%%%%%%%%
\subsection{Kicker}
%\RK{descripcion del pateador flyback}[Ruben: si existe una imagen mejor del circuito (vector), hay que actualizarla]
\PR{Mostrar la nueva placa de pateo y explicar técnicamente cómo funciona. Incorporar figura para comparar cambios.}

In the former version of our robots, the circuit of the kicking system was a Boost converter as shown in the Figure \ref{fig:old_boost}. This converter raises the voltage from the battery to 120 [V], but the main disadvantage of the circuit is the charging time, taking about 6 seconds to reach the final value. Given the dynamics and speed of the game, this charging time is excessive.

\begin{figure}[h]
    \centering
    \includegraphics[width=0.5\textwidth]{Images/foto_boost.jpg}
    \caption{Old Boost circuit}
    \label{fig:old_boost}
\end{figure}

This year, the kicking system has been modified and now consists in a flyback circuit as shown in the Figure \ref{fig:kicker}. This circuit uses the Chip LT3751 which is a charger controller with regulation. The flyback circuit implements a turn ratio of 1:10 (primary:secondary) and raises the voltage from 14.8 [V] to 220 [V] on two capacitors of 1200[$\mu$F], each in less than a half second. The main advantage is that the voltage setpoint can be regulated to kick with different intensities.
%\TR{Esta imagen esta repetida en el TDP 2018. Si no hay cambios en el diseño no deberiamos ponerlo.}

\begin{figure}[h]
    \centering
    \includegraphics[scale=0.4]{Images/kicker_spice.png}
    \caption{Kicker Circuit}
    \label{fig:kicker}
\end{figure}


%%%%%%%%%%%%%%%%%%%%%%%%%%%%%%%%%%%%%%%%%%%%%%%%%%%%%%%%%%%%%%%%%%%%%%%%%%%%%%%%%%%%%%%%%%%%%%%
%                                 Bibliografy                                                 %
%%%%%%%%%%%%%%%%%%%%%%%%%%%%%%%%%%%%%%%%%%%%%%%%%%%%%%%%%%%%%%%%%%%%%%%%%%%%%%%%%%%%%%%%%%%%%%%
\bibliographystyle{plain}
{
    \renewcommand{\clearpage}{} 
    \bibliography{biblio.bib}
}



%%%%%%%%%%%%%%%%%%%%%%%%%%%%%%%%%%%%%%%%%%%%%%%%%%%%%%%%%%%%%%%%%%%%%%%%%%%%%%%%%%%%%%%%%%%%%%%
%                             Acknowledgments                                                 %
%%%%%%%%%%%%%%%%%%%%%%%%%%%%%%%%%%%%%%%%%%%%%%%%%%%%%%%%%%%%%%%%%%%%%%%%%%%%%%%%%%%%%%%%%%%%%%%
\section*{Acknowledgments}

Sysmic Robotics would like to acknowledge the previous team members that helped us to get here and to all the people that support and help us in any way. We could not have made it alone.

% that's all folks
\end{document}